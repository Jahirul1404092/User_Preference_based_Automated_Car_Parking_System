\section{Conclusion}
A low cost smart car parking system we developed based on internet of thins technology. We discussed current condition, previous works, motivation behind our thesis, objectives and contribution of our works. We discussed on various hardware components we used and software modules we developed. We used googles firebase realtime database. We tried to make our system secured and database private thus anyone can not access to the system. In our system sensors are sensing device, led is the indicatior, arduino is the processor, wifi module is the gateway to send and receive data from device to database and database to device. Android application is the interaction media of user with this system. We discussed on the principle of our system. We evaluated the performance of our developed system.

\section{Limitation and Future Work}
We developed this smart car parking system as an experimental setup. we made a prototype of it. We developed an android application. We tried to show the feasibility of this concept. We found this concept is feasible and this system can be developed in real time. We found some latency in signal transferring due to slow internet speed. So we should use more fast internet thus data transferring latency could be minimized. We found some errors and fluctuation when multiple users tried to park cars. This is because as the processor we used arduino. And arduino can not do parallel processing. It processes sequentially. Again we used low cost sensor which is not too good to work at high load. So in the real field industrial graded sensor should be used and higher quality processor like STM32 or raspberry pi has to be used. We used a dummy money account for payment because to add this type of account company wants business information to open a merchant online bank account. As we are not businessmen, we could not be able to provide that information. When this system will be implemented in real life, the real merchant account has to be used. 